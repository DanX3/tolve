%%%%%%%%%%%%%%%%%%%%%%%%%%%%%%%%%%%%%%%%%
% Thin Sectioned Essay
% LaTeX Template
% Version 1.0 (3/8/13)
%
% This template has been downloaded from:
% http://www.LaTeXTemplates.com
%
% Original Author:
% Nicolas Diaz (nsdiaz@uc.cl) with extensive modifications by:
% Vel (vel@latextemplates.com)
%
% License:
% CC BY-NC-SA 3.0 (http://creativecommons.org/licenses/by-nc-sa/3.0/)
%
%%%%%%%%%%%%%%%%%%%%%%%%%%%%%%%%%%%%%%%%%

%----------------------------------------------------------------------------------------
%	PACKAGES AND OTHER DOCUMENT CONFIGURATIONS
%----------------------------------------------------------------------------------------

\documentclass[a4paper, 11pt]{article} % Font size (can be 10pt, 11pt or 12pt) and paper size (remove a4paper for US letter paper)

\usepackage[protrusion=true,expansion=true]{microtype} % Better typography

\usepackage[utf8]{inputenc}


\usepackage{mathpazo} % Use the Palatino font
\usepackage[T1]{fontenc} % Required for accented characters

\makeatletter
\renewcommand\@biblabel[1]{\textbf{#1.}} % Change the square brackets for each bibliography item from '[1]' to '1.'
\renewcommand{\@listI}{\itemsep=0pt} % Reduce the space between items in the itemize and enumerate environments and the bibliography

\renewcommand{\maketitle}{ % Customize the title - do not edit title and author name here, see the TITLE block below
\begin{flushright} % Right align
{\LARGE\@title} % Increase the font size of the title

\vspace{50pt} % Some vertical space between the title and author name

{\large\@author} % Author name
\\\@date % Date

\vspace{40pt} % Some vertical space between the author block and abstract
\end{flushright}
}

%----------------------------------------------------------------------------------------
%	TITLE
%----------------------------------------------------------------------------------------

\title{\textbf{Relazione progetto ChatFe} }

\author{\textsc{Daniele Tolomelli \\ Simone Venturelli}} % Author

\date{\today} % Date

%----------------------------------------------------------------------------------------

\begin{document}

\maketitle % Print the title section


\section*{Introduzione}

Nella realizzazione di questo progetto abbiamo scelto di implementare prima una serie di funzioni di utilità, non vincolate direttamente al sorgente principale, ma disegnate sulle sue esigenze, in modo da semplificare successivamente il lavoro. Questo sia per la leggibilità del codice sia per garantire l'inserimento di porzioni di codice precedentemente testato, al fine di restringere il campo nella ricerca di eventuali errori. Abbiamo cercato di usare il più possibile funzioni di sistema, sempre per ottenere codice più veloce e testato possibile.\\ 
Generalmente nel disegnare le funzioni abbiamo cercato di far passare all'utente il minor numero possibile di argomenti, in modo da ridurre ulteriori errori alla chiamata delle funzioni. Un esempio soggetto a questo stile implementativo sono le funzioni di \texttt{message.c} le quali, pur rischiando di suonare ridondanti, spiegano nel nome il loro intento e limitano l'utente a fornire soltanto l'effettivo contenuto dei messaggi. 
Abbiamo ordinato queste funzioni in diversi file .c suddivise per categorie di utilizzo, e abbiamo raccolto i prototipi delle rispettive funzioni nel file \texttt{utils.h}.\\

\begin{itemize}
	\item \texttt{hdata.c}\\
	Funzioni di gestione della hash table: salvataggio e caricamento da file e conversione dell'elemento in stringa e viceversa. La funzione \texttt{getDataFrom()} restituendo il puntatore all'elemento allegerisce semplicemente la sintassi al sorgente chiamante.
	\item  \texttt{log.c}\\
	Funzioni di gestione del file di log: scrittura dell'accesso di un utente al server, scrittura di un messaggio da parte di un utente e scrittura di un errore. A queste funzioni ha accesso unicamente il server.
	\item  \texttt{stringList.c}\\
	Funzioni che gestiscono un tipo di dato astratto \texttt{StringList} che usiamo per memorizzare gli utenti attualmente connessi. Le funzioni sono di inserimento, cancellazione, verifica e listing degli elementi.
	\item  \texttt{message.c}\\
	Funzioni di comunicazione tra Server e Client. Le iniziali nel nome della funzione indicano la direzione della comunicazione, prima lettera per il mittente e seconda per il destinatario.
	\item  \texttt{ringBuffer.c}\\
	Funzioni che gestiscono le comunicazioni tra thread attraverso un buffer circolare. Implementando queste funzioni si sfoltisce notevolmente il codice nel sorgente principale, delegando tutto il lavoro a queste funzioni. Queste scrivono e leggono esattamente la lunghezza del messaggio, evitando cosi scritture di caratteri vuoti, ottimizzando così lettura e scrittura. 
	\item  \texttt{misc.c}\\
	Raccolta di varie funzioni, come \texttt{timestamp()} per l'inizializzazione del file di log e funzioni di marshalling. La funzione \texttt{cmdmatcher()} traduce il comando inserito dal client in un corrispettivo intero, in modo da poter scrivere il sorgente del client attraverso una struttura \texttt{switch}, adatta alle numerose richieste del client.
\end{itemize}

\section*{Server}
Non facciamo notare nulla di particolare nell'implementazione del server poiché rispecchia nei tratti generali la consegna. Alcune scelte implementative sono l'uso della \texttt{calloc()} al posto della suggerita combinazione \texttt{malloc()} - \texttt{bzero()} per compattare le inizializzazioni. Riserviamo la \texttt{bzero()} solo per reinizializzare i dati all'inizio di un nuovo ciclo, in modo da mantenere la posizione in memoria e non occupare ulteriore spazio.

\section*{Client}
La funzione \texttt{reglog()} in realtà esegue solo la registrazione, ma il controllo passa immediatamente alla fase di login, che viene quindi eseguito in automatico. In questo modo si sovraccaricano le operazioni di comunicazione, dovendo ripetere un passaggio, riuscendo però a snellire il codice. Essendo un operazione che non viene eseguita molto frequentemente consideriamo accettabile questo compromesso. 
	

\end{document}